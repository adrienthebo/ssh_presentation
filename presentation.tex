\documentclass{beamer}

\mode<presentation> {\usetheme{Antibes} }

\usepackage{times}

\title{SSH}

\author{
Finch \\
The CAT
}

\AtBeginSection[]
{
  \begin{frame}<beamer>
    \frametitle{Outline}
    \tableofcontents[currentsection]
  \end{frame}
}

\begin{document}

\begin{frame}
  \titlepage
\end{frame}

\section{Introduction}

\begin{frame}
  \frametitle{Secure Shell - What is it?}
  \begin{itemize}
    \item Network protocol designed to allow exchange of data between two systems
    \item Designed with security in mind - secure connection over insecure medium
    \item Co-opted to do more than just shells
  \end{itemize}
\end{frame}

\begin{frame}
  \frametitle{Before SSH}
  \begin{itemize}
    \item The roles that SSH serve today were done by other means
    \begin{itemize}
      \item File transfer - ftp
      \item Remote shell - rlogin
      \item Remote graphics - XDMCP
    \end{itemize}
    \item All of these are unencrypted
    \pause
    \item Why is this a problem?
  \end{itemize}
\end{frame}

\begin{frame}
  \frametitle{The Mysterious Tale of phantomd}
He had discovered an account named Anitha on Sunday, June 9. If he had
immediately stomped the account, he might have been able to wash his hands of
the problem. Instead he had copied the (freely available) source code for
Telnet from a computer at Berkeley on Monday. In a daylong bout of programming,
he doctored Telnet to record all incoming and outgoing communications from the
Anitha account. After installing the bugged Telnet, Janaka knocked off for some
sleep. He came to the office Tuesday to find a pile of printout on his desk.
\end{frame}

\begin{frame}
  \frametitle{We like encryption}
  \begin{itemize}
    \item In the day of ``Free Public Wifi,'' encryption is key
    \item Encryption protects your banking information
    \item Encryption protects your email
    \item You probably want encryption for your {\bf ROOT SHELL}
  \end{itemize}
\end{frame}

\section{An aside on cryptography}

\begin{frame}
  \frametitle{How does cryptography work?}
  \begin{itemize}
    \item There are two prominent types of cryptography
    \item Symmetric cryptography
    \item Asymmetric cryptography
  \end{itemize}
\end{frame}

\begin{frame}
  \frametitle{A brief explanation of symmetric cryptography}
  \begin{itemize}
    \item Symmetric cryptography is any cryptographic scheme that uses the same key to encrypt or decrypt the payload
    \item Both sides of the communication use the same shared secret to communicate
    \item Examples:
      \begin{itemize}
	\item AES (Advanced Encryption Standard)
	  \begin{itemize}
	    \item A device that could check a billion billion AES keys per second would in theory require about $30^{51}$ years to exhaust the 256-bit key space.
	  \end{itemize}
	\item Blowfish
	\item XOR
    \end{itemize}
    \pause
  \item Symmetric algorithms are generally very fast, but have their limits
  \item Since the same key is used for encryption and decryption, how do you start communication with another party?
  \end{itemize}
\end{frame}

\begin{frame}
  \frametitle{A brief explanation of asymmetric cryptography}
  \begin{itemize}
    \item Asymmetric cryptography is any cryptographic scheme that uses different keys to encrypt and decrypt conversations 
    \item Examples:
      \begin{itemize}
	\item RSA (Rivest, Shamir, Adleman)
	\item DH (Diffie, Hellman)
    \end{itemize}
    \pause
  \item Asymmetric algorithms are generally fairly slow, but allow for initiation secure communications between two parties
  \item Thus, asymmetric algorithms can be used to start a symmetric cryptographic session
    \begin{itemize}
      \item Secure and easy initiation of communication
      \item High rate of transfer
    \end{itemize}
  \end{itemize}
\end{frame}

\section{Using SSH}

\begin{frame}
  \frametitle{SSH Keys}
  \begin{itemize}
    \item SSH key pairs are ``key'' to how ssh works
    \pause
    \item One pair of keys for each host
    \item This allows hosts to verify each other's identity, to avoid man in the middle attacks
  \end{itemize}
\end{frame}

\begin{frame}
  \frametitle{SSH User Keys}
  \begin{itemize}
    \item User key pairs are also possible, to avoid using password authentication
    \item Public key is copied to remote host, private key is kept secure on the local host
    \item It is possible to password protect your private key, for an added layer of protection
    \item SSH keypairs are effectively impossible to brute force - MUCH more secure than passwords.
      \pause
    \item Having many different keypairs for different machines is ideal
    \item One key for you github account
    \item One for your laptop
    \item One for you desktop
    \item One for CECS general access
    \item One for highly secure machines
      \pause
    \ldots
    \item That's a lot of passwords to type and memorize
  \end{itemize}
\end{frame}

\begin{frame}
  \frametitle{SSH agent}
  \begin{itemize}
    \item ssh-agent will store your decrypted private key in memory for each key added
    \item Need to connect to a series of host in rapid succession? No problem!
  \end{itemize}
\end{frame}

\begin{frame}
  \frametitle{SSH - not just for shells}
  \begin{itemize}
    \item agent forwarding
    \item compression
    \item socks proxy
    \item port forwarding, local and remote
    \item X forwarding
  \end{itemize}
\end{frame}

\begin{frame}
  \frametitle{Configuring SSH}
  \begin{itemize}
    \item ssh\_config
    \item authorized\_keys
    \item known\_hosts
  \end{itemize}
\end{frame}

\end{document}
